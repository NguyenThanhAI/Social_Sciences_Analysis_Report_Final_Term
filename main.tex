\documentclass[14pt, a4paper]{article}
\usepackage{minitoc}
\usepackage[left=3.00cm, right=2.5cm, top=2.00cm, bottom=2.00cm]{geometry}
\usepackage{amsmath}
\usepackage{amssymb}
\usepackage{amsthm}
\usepackage{mathtools}
\usepackage{graphicx}
%\usepackage{algpseudocode}
%\usepackage{algorithm}
\usepackage[ruled,vlined,linesnumbered]{algorithm2e}
\usepackage{blindtext}
\usepackage{setspace}
\usepackage[utf8]{inputenc}
\usepackage[utf8]{vietnam}
\usepackage[center]{caption}
\usepackage[shortlabels]{enumitem}
\usepackage{fancyhdr} % header, footer
\usepackage{hyperref} % loại bỏ border với mục lục và công thức
\usepackage[nonumberlist, nopostdot, nogroupskip]{glossaries}
\usepackage{glossary-superragged}
\usepackage{tikz,tkz-tab}
\usepackage{pythonhighlight}
\usepackage{float}
\setglossarystyle{superraggedheaderborder}



\pagestyle{fancy}
%\usepackage[style=numeric,sortcites]{biblatex}
%\addbibresource{ref.bib}
%\usepackage[numbers]{natbib}
\usepackage{indentfirst}
\usepackage[natbib,backend=biber,style=ieee, sorting=ynt]{biblatex}

\usepackage{caption}
\usepackage{subcaption}

\usepackage{subfiles} % Best loaded last in the preamble

\usepackage{makecell}

\renewcommand\theadalign{bc}
\renewcommand\theadfont{\bfseries}
\renewcommand\theadgape{\Gape[4pt]}
\renewcommand\cellgape{\Gape[4pt]}


\bibliography{ref.bib}

\graphicspath{{./figures/}}

\fancyhf{}
%\rhead{\textbf{Môn học: Các phương pháp thống kê hiện đại trong nghiên cứu Xã hội học}}
\lhead{\textbf{GVHD: TS. Trịnh Quốc Anh}}
\rfoot{\thepage}
\lfoot{\textbf{Học viên thực hiện: Nguyễn Chí Thanh - 21007925}}
\renewcommand{\headrulewidth}{0.4pt}
\renewcommand{\footrulewidth}{0.4pt}
%
\numberwithin{equation}{section}
%\numberwithin{algorithm}{section}
\numberwithin{figure}{section}
%
\setlength{\parindent}{0.5cm}
%
%\setcounter{secnumdepth}{3} % Cho phép subsubsection trong report
%\setcounter{tocdepth}{3} % Chèn subsubsection vào bảng mục lục

%\newtheorem{dl}{Định lý}
%\newtheorem{md}{Mệnh đề}
%\newtheorem{bd}{Bổ đề}
%\newtheorem{dn}{Định nghĩa}
%\newtheorem{hq}{Hệ quả}

%\newtheorem{baitap}{Bài tập}
%\newtheorem*{loigiai}{Lời giải}

%\numberwithin{dl}{section}
%\numberwithin{md}{section}
%\numberwithin{bd}{section}
%\numberwithin{dn}{section}
%\numberwithin{hq}{section}

\setlength{\parindent}{0cm}

\newtheorem{dl}{Định lý}
\newtheoremstyle{sltheorem}
{}                % Space above
{}                % Space below
{\normalfont}        % Theorem body font % (default is "\upshape")
{}                % Indent amount
{\bfseries}       % Theorem head font % (default is \mdseries)
{.}               % Punctuation after theorem head % default: no punctuation
{ }               % Space after theorem head
{}                % Theorem head spec
\theoremstyle{sltheorem}
\newtheorem{baitap}{Bài tập}
\newtheoremstyle{soltheorem}
{}                % Space above
{}                % Space below
{\normalfont}        % Theorem body font % (default is "\upshape")
{}                % Indent amount
{\bfseries}       % Theorem head font % (default is \mdseries)
{.}               % Punctuation after theorem head % default: no punctuation
{\newline}               % Space after theorem head
{}                % Theorem head spec
\theoremstyle{soltheorem}
\newtheorem*{loigiai}{Lời giải}

\onehalfspacing

\begin{document}
\begin{titlepage}

    \newcommand{\HRule}{\rule{\linewidth}{0.5mm}} % Defines a new command for the horizontal lines, change thickness here

    \center % Center everything on the page

    %----------------------------------------------------------------------------------------
    %	HEADING SECTIONS
    %----------------------------------------------------------------------------------------
    \textsc{\LARGE Đại học Quốc Gia Hà Nội}\\[0.5cm]
    \textsc{\LARGE Trường đại học Khoa học tự nhiên}\\[0.5cm] % Name of your university/college
    \textsc{\LARGE Khoa Toán - Cơ - Tin học}\\[0.5cm]

    \includegraphics[scale=0.2]{HUS-logo.jpg}\\[0.5cm]

    \textsc{\Large Chuyên ngành: Khoa học dữ liệu}\\[0.5cm] % Major heading such as course name


    %----------------------------------------------------------------------------------------
    %	TITLE SECTION
    %----------------------------------------------------------------------------------------

    \HRule \\[0.4cm]
    { \huge \bfseries Bài tập cuối kỳ}\\[0.4cm] % Title of your document
    \HRule \\[1.5cm]

    \textsc{\Large Môn học: Các phương pháp thống kê hiện đại \\ trong nghiên cứu Xã hội học}\\[1cm] % Minor heading such as course title


    \textsc{\Large Đề tài: }\\[1cm]


    %----------------------------------------------------------------------------------------
    %	AUTHOR SECTION
    %----------------------------------------------------------------------------------------
    \begin{minipage}{0.4\textwidth}
        \begin{flushleft} \large
            \emph{Giảng viên hướng dẫn:} \\
            TS. Trịnh Quốc Anh % Supervisor's Name
        \end{flushleft}
    \end{minipage}\\[0.5cm]

    \begin{minipage}{0.4\textwidth}
        \begin{flushleft} \small
            \emph{Nhóm học viên thực hiện:}\\
            Nguyễn Chí Thanh \\
            MSHV: 21007925 \\ % Your name
            Nguyễn Đức Thịnh \\
            MSHV: 21007923 \\
            Vũ Ngọc Đại \\
            MSHV: 21007977 \\
            Vũ Minh Hưng \\
            MSHV: 21007973 \\
            Nguyễn Tiến Huy \\
            MSHV: 21007974\\
            Vũ Hải Bằng \\
            MSHV: 21007932\\
            Lớp: Khoa học dữ liệu - K4
        \end{flushleft}
    \end{minipage}


    % If you don't want a supervisor, uncomment the two lines below and remove the section above
    %\Large \emph{Author:}\\
    %John \textsc{Smith}\\[3cm] % Your name

    %----------------------------------------------------------------------------------------
    %	DATE SECTION
    %----------------------------------------------------------------------------------------

    % I don't want day because it is English
    % {\large \today}\\[2cm] % Date, change the \today to a set date if you want to be precise

    %----------------------------------------------------------------------------------------
    %	LOGO SECTION
    %----------------------------------------------------------------------------------------

    %\includegraphics{logo/rsz_3logo-khtn.png}\\[1cm] % Include a department/university logo - this will require the graphicx package

    %----------------------------------------------------------------------------------------

    \vfill % Fill the rest of the page with whitespace

\end{titlepage}

\nocite{*}


\textbf{Đề bài: }Từ dữ liệu đã cho, hãy đưa ra mô hình dự báo về hình thức làm việc "wrkstat" của một người Mỹ trưởng thành vào năm 2016. 
Những yếu tố sử dụng mạng xã hội là "snapchat" và "instagram" có quyết định đến hình thức làm việc của một người hay không? 
Đưa ra các giải thích chi tiết về quá trình xây dựng, đánh giá, và biện luận mô hình dự báo.


    % \begin{enumerate}
    %     \item Phân tích dữ liệu
        
    %     \item Cơ sở lý thuyết
    %     \begin{itemize}
    %         \item Principle Component Analysis
            
    %         \subfile{files/pca.tex}

    %         \item Mô hình Multinomial Logistic Regression
            
    %         \subfile{files/multinomial_logistic_regression.tex}

    %         \item Mô hình Bayesian Multinomial Logistic Regression
            
    %         \subfile{files/bayesian_multinomial_logistic_regression.tex}

    %         \item Mô hình rừng ngẫu nhiên (Random Forest)

    %         \subfile{files/random_forest_classifier.tex}

    %     \end{itemize}
    % \end{enumerate}

    \cleardoublepage
    \pagenumbering{gobble}
    \tableofcontents
    \newpage
    \listoffigures
    \cleardoublepage
    \pagenumbering{arabic}

    \section{Thống kê mô tả và phân tích tập dữ liệu}
    
    \subfile{files/huy-dai-bang.tex}

    \subfile{files/data_analysis_non_null.tex}

    \subfile{files/data_analysis_with_null.tex}

    \section{Xây dựng mô hình}

    \subfile{files/models/models.tex}

    \subfile{files/models/logistic_non_null.tex}
    
    \subfile{files/models/logistic_with_null.tex}
    
    \subfile{files/models/random_forest_non_null.tex}
    
    \subfile{files/models/random_forest_with_null.tex}
    
    \subfile{files/models/adaboost_non_null.tex}

    \subfile{files/models/adaboost_with_null.tex}
    
    \subfile{files/models/xgboost_non_null.tex}

    \subfile{files/models/xgboost_with_null.tex}

    \subfile{files/models/mlp_non_null.tex}

    \subfile{files/models/mlp_with_null.tex}

    \subfile{files/models/bayesian_non_null.tex}

    \subfile{files/models/bayesian_with_null.tex}

    \subfile{files/models/models_summary.tex}

    \subfile{files/summary.tex}


\newpage
\printbibliography[title={TÀI LIỆU THAM KHẢO}]

\newpage

\appendix

\section{Cơ sở lý thuyết}

\subsection{Principle Component Analysis}

\subfile{files/pca.tex}

\subsection{Mô hình Multinomial Logistic Regression}

\subfile{files/multinomial_logistic_regression.tex}

\subsection{ Mô hình Bayesian Multinomial Logistic Regression}

\subfile{files/bayesian_multinomial_logistic_regression.tex}

\subsection{Mô hình rừng ngẫu nhiên (Random Forest)}

\subfile{files/random_forest_classifier.tex}

\subsection{Boosting}
    
\subfile{files/boosting.tex}

\subsection{XGBoost}

\subfile{files/xgboost.tex}

\newpage
\section{Chương trình chạy}

Link github repository: \url{https://github.com/NguyenThanhAI/Social_Science_Analysis_Code_Final_Term}

Tên các file chương trình và chức năng:

\begin{itemize}
    \item new\_stats\_non\_null.ipynb: Thống kê mô tả và phân tích dữ liệu trên tập dữ liệu chỉ bao gồm các quan sát có cột "emailtotal" không phải giá trị null.
    \item new\_stats\_with\_null.ipynb: Thống kê mô tả và phân tích dữ liệu trên tập dữ liệu tập dữ liệu chỉ bao gồm các quan sát có cột "emailtotal" chỉ là giá trị null.
    \item multinomial\_logistic\_regression.ipynb: Thực hiện mô hình Multinomial Logistic Regression
    \item random\_forest.ipynb: Thực hiện mô hình Random Forest
    \item boosting.ipynb: Thực hiện mô hình AdaBoost 
    \item extreme\_gradient\_boosting.ipynb: Thực hiện mô hình XGBoost
    \item mlp\_deep\_learning.ipynb: Thực hiện mô hình Multi layer Perception Deep Learning
    \item bayesian\_multinomial\_logistic\_regression: Thực hiện mô hình Bayesian Multinomial Logistic Regression
    \item Impact\_of\_using\_snapchat\_and\_instagram.ipynb: Thực hiện kiểm định yếu tố sử dụng mạng xã hội là "snapchat" và "instagram" có quyết định đến hình thức làm việc của một người hay không
\end{itemize}
\end{document}